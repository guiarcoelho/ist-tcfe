\section{Simulation analysis}
\label{sec:simulation}

\subsection{Simulating the AC/DC converter for 10 periods}
The first step to this laboratory assignment was to make a simulation of a simple AC/DC converter in NGSpice. The circuit uses an ideal transformer, using a current controlled voltage source as well as a voltage controlled current source, and also an envelope detector and a voltage regulator. \par
Altough, because od the fact that the ideal transformer creates a new voltage and current with the same frequency as the original, instead of controlled sources, a simple sinusoidal voltage source would be enough.
A simulation of the AC/DC converter was done for 10 periods and all the analysis were made measuring on a $5e-5$ step with the objective of evaluating at least 1000 points during the 10 periods. The frequency of the AC sourcec was used to know the period and then multiplied by 10 with the goal of getting the total time, in order to be able to calculate this step. After that we divided the total time by 1000 points and made the step even smaller than that with the goal of making sure it had more than 1000 points but not too small so that the program ran poorly. \par


\subsection{Output voltage level}
We used NGSpice to measure the average output voltage and using a transient analysis we plotted both the average and the signal of the output voltage in the same graph.

\begin{table}[H]
  \centering
  \begin{tabular}{|l|r|}
    \hline    
    {\bf Name} & {\bf Value [V]} \\ \hline
    \input{average_tab}
  \end{tabular}
  \label{tab:average}
\end{table}

\begin{figure}[H] \centering
\includegraphics[width=0.7\linewidth]{../sim/transient1.pdf}
\caption{Plot of the average and the signal of the Output Voltage.}
\label{fig:transient1}
\end{figure}

\subsection{Output of the Envelope Detector and voltage Regulator circuits}
The output voltages of the Envelope Detector and the Voltage Regulator circuits were plotted and put each in a different graph as well as a graph with both voltages plotted. \par

\begin{figure}[H] \centering
\includegraphics[width=0.7\linewidth]{../sim/transient2.pdf}
\caption{Envelope Detector Output Voltage.}
\label{fig:transient2}
\end{figure}

\begin{figure}[H] \centering
\includegraphics[width=0.7\linewidth]{../sim/transient3.pdf}
\caption{Voltage Regulator Output Voltage.}
\label{fig:transient3}
\end{figure}

\begin{figure}[H] \centering
\includegraphics[width=0.7\linewidth]{../sim/transient4.pdf}
\caption{Envelope Detector and Voltage Regulator Output Voltages.}
\label{fig:transient4}
\end{figure}

\subsection{Output voltage ripple}
We then used NGSpice to measure the output voltage ripple, that is the difference between the maximum and the minimum values of the signal. \par

\begin{table}[H]
  \centering
  \begin{tabular}{|l|r|}
    \hline    
    {\bf Name} & {\bf Value [V]} \\ \hline
    \input{ripple_tab}
  \end{tabular}
  \label{tab:ripple}
\end{table}

\subsection{$v_0 - 12$ plot}
We plotted $v_0 - 12$, which is the output AC component plus the DC deviation, and calculated the deviation, using the mean value. \par

\begin{figure}[H] \centering
\includegraphics[width=0.7\linewidth]{../sim/transient5.pdf}
\caption{Output AC component + DC deviation.}
\label{fig:transient5}
\end{figure}

\begin{table}[H]
  \centering
  \begin{tabular}{|l|r|}
    \hline    
    {\bf Name} & {\bf Value [V]} \\ \hline
    \input{meanv012_tab}
  \end{tabular}
  \label{tab:meanv012}
\end{table}
