\section{Theoretical Analysis}
\label{sec:analysis}

In the theoretical analyses we apply Kirchoff's Laws and diodes equatins to calculate the obtained results. It's important to differ the Evelope Detector (whose function is to restrict voltage's amplitude) from the Voltage Regulator (whose purpose is to decrease the ripple).
\subsection{Envelope detector}
The envelope detector is composed of  a resistance and  capacitor in parallel. It decreases the ripple according to the following expression:\par
\begin{equation}
    v_0(t) = Acos(\omega t_{off})^{\frac{-t+t_{off}}{RC}}
\end{equation}\par
Where: \par
\begin{itemize}
  \item A - Amplitude
  \item $\omega$ - angular frequency
  \item R - Resistance, constant obtained using the following expression:
\end{itemize}
\begin{equation}
    R = R_3 + rd_n 
\end{equation}\par
\begin{itemize}
  \item $R_3$ - Resistence in serie
  \item $rd_n$ - Resistence in all diodes,  wich currespondes to 23*$r_d$ (the resistance in each diode)
\end{itemize}
\begin{itemize}
  \item C - Capacitance
  \item $t_{off}$ - Constant obtained using the following expression:
\end{itemize}
\begin{equation}
    t_{off}=\frac{1}{\omega}arctan(\frac{1}{\omega RC})
\end{equation}\par

\begin{figure}[H] \centering
\includegraphics[width=0.7\linewidth]{../mat/retified.pdf}
\caption{Rectified Signal.}
\label{fig:rectified}
\end{figure}

\begin{figure}[H] \centering
\includegraphics[width=0.7\linewidth]{../mat/envelope.pdf}
\caption{Envelope Detector Output Voltage.}
\label{fig:envelopeth}
\end{figure}

\subsection{Voltage Regulator}
The voltage regulator is composed by 19 diodes in series that will impose the 12V voltage. The resistance in series will decrease the amplitude.\par
The expressions used to do this were the following ones:\par
Sinusoidal part from envelope:\par
\begin{equation}
    v_{sr} = v_O - V_s
\end{equation}\par
And then, we have: \par
\begin{equation}
    vO_r = (\frac{rd_n}{rd_n + R_3})v_{sr}
\end{equation}\par
The expression used to obtain the vltage ripple was: \par
\begin{equation}
    v_{ripple} = maximum(V_{dc})-minimum(V_{dc})
\end{equation}\par
Where: \par
\begin{equation}
    V_{dc} = 21V_{on} + vO_r
\end{equation}\par
The results are shown below:\par

\begin{figure}[H] \centering
\includegraphics[width=0.7\linewidth]{../mat/outputdc.pdf}
\caption{DC Output Signal.}
\label{fig:outputdc}
\end{figure}

\begin{figure}[H] \centering
\includegraphics[width=0.7\linewidth]{../mat/v012.pdf}
\caption{Output Signal - 12 (deviation).}
\label{fig:v012}
\end{figure}

\begin{table}[H]
  \centering
  \begin{tabular}{|l|r|}
    \hline    
    {\bf Name} & {\bf Value [V]} \\ \hline
    \input{../mat/ripple_octave}
  \end{tabular}
  \label{tab:ripple}
\end{table}

\begin{figure}[H] \centering
\includegraphics[width=0.7\linewidth]{../mat/ripple.pdf}
\caption{Ripple.}
\label{fig:ripplegraph}
\end{figure}
