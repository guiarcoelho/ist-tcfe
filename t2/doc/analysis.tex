\section{Theoretical Analysis}
\label{sec:analysis}

In this section we are going to apply the methods as told in the exercise of the laboratory to be able to compare the results with the simulation results. As it is required in the ngspice we will add a voltage 0 source to the circuit, placing it betwin resistor 6 and node 7, creating a node 4 in betwin the sorce and the resistor.

\begin{figure}[h] \centering
\includegraphics[width=0.6\linewidth]{circuitong.pdf}
\caption{Circuit in study}
\label{fig:circuit_t2}
\end{figure}



\subsection{Analysis for $t<0$ - Nodal Method} 
For $t<0$ the capacitor has no current thrue it, so it works as an open circuit. Aplying the Nodes Method we can find the following equations. From the ground placement on the circuit we get equation (\ref{eq:n1}), from the 0 voltage source we get equation (\ref{eq:n2}), from the information given about the dependent voltage source we get equation (\ref{eq:n3}) and from the independent voltage source we get equation (\ref{eq:n4}). Then, by aplying KCL to the nodes 2 (\ref{eq:n5}), 3 (\ref{eq:n6}), 6 (\ref{eq:n8}), 7 (\ref{eq:n9}) and 5, considering that the dependent voltage source doesn't change the current, (\ref{eq:n7}).\par
\begin {equation}
V_0 = 0
\label{eq:n1}
\end{equation}
\begin {equation}
V_4 = V_7
\label{eq:n2}
\end{equation}
\begin {equation}
	V_5 - V_8 = K_d \frac{V_0 - V_4}{R_6}
	\label{eq:n3}
\end{equation}
\begin {equation}
	V_1 - V_0 = V_s
	\label{eq:n4}
\end{equation}
\begin {equation}
	\frac{V_2-V_1}{R_1} + \frac{V_2 - V_5}{R_3} + \frac{V_2 - V_3}{R_2} = 0
	\label{eq:n5}
\end{equation}
\begin {equation}
	\frac{V_3-V_2}{R_2} - K_b(V_2-V_5)  = 0
	\label{eq:n6}
\end{equation}
\begin {equation}
	\frac{V_5-V_2}{R_3} + \frac{V_5-V_0}{R_4} + \frac{V_5-V_6}{R_5} + \frac{V_8-V_7}{R_7}= 0
	\label{eq:n7}
\end{equation}
\begin {equation}
	\frac{V_6-V_5}{R_5} + K_b(V_2-V_5)  = 0
	\label{eq:n8}
\end{equation}
\begin {equation}
	\frac{V_4-V_0}{R_6} + \frac{V_7 - V_8}{R_7} = 0
	\label{eq:n9}
\end{equation}



\begin{table}[H]

  \centering 
  \begin{tabular}{|l|r|}
    \hline    
    {\bf Name} & {\bf Nodal method}\\ \hline
    \input{nodal_tab}
  \end{tabular}
  \vspace{10px}
  \caption{A variable that starts with "@" is of type {\em current}
    and expressed in Ampere (A); all the other variables that start with a "V" are of the type {\it voltage} and expressed in
    Volt (V).}
  \label{tab:theoretical}
  
\end{table}



\subsection{Equivalent resistor as seen from the capacitor terminals}
To obtain the equivalent resistence as viewd by C we need to remove the independt voltage source, by replacing it with a short circuit ($V_s=0$). Because of the dependent source we also need to  replace the capacitor with a voltage source $V_x = V_6-V_8$. We use the $V_6$ and $V_8$ from the previous section beacause the voltage drop at the terminals of the capacitor needs to be a continuous function (there can not be an energy
discontinuity in the capacitor).\par
Now we perform an other nodal analysis to figure out the current $I_x$ supplied by the immaginary voltage source $V_x$, to then be able to find the equivalent resistance ($R_eq = V_x/I_x$).The equations considered for these calculations were \ref{eq:n1}, \ref{eq:n2}, \ref{eq:n3}, \ref{eq:n4}, \ref{eq:n5}, \ref{eq:n6} and the following:


\begin {equation}
	\frac{V_1-V_2}{R_1} + \frac{V_0-V_4}{R_6} + \frac{V_0-V_5}{R_4} = 0
	\label{eq:eq7}
\end{equation}
\begin {equation}
	K_b(V_2-V_5) + \frac{V_6-V_5}{R_5} + I_x  = 0
	\label{eq:eq8}
\end{equation}
\begin {equation}
	\frac{V_4-V_0}{R_6} + \frac{V_7 - V_8}{R_7} = 0
	\label{eq:eq9}
\end{equation}
\begin {equation}
	V_x = V_6 - V_8
	\label{eq:eq10}
\end{equation}
\newpage

\begin{table}[H]

  \centering
  \begin{tabular}{|l|r|}
    \hline    
    {\bf Name} & {\bf Theoretical values }\\ \hline
    \input{req_tab}
  \end{tabular}
  \vspace{10px}
  \caption{A variable that starts with a "V" is of type {\it voltage} and expressed in
    Volt (V). The variable $R_{eq}$ is expressed in Ohm and the variable $\tau$ is expressed in seconds }
  \label{tab:equivalent resistor}
\end{table}

%point 3
\subsection{Natural solution for V6}

To calculate the natural solution for V6 we first need to simplify the circuit by removing all independent sources and aplying KVL. We then get the natural solution by the equation $V_{6n}(t)=Ae^{\frac{-t}{\tau}}$.

The process to get to this point is explained in the theoretical class nº3. $\tau$ is the time constant, calculated with the equivalent resistance from the previous section ($\tau=R_{eq}*C$) and A is taken from the boundry conditions also calculated in the last section ($t=0, A=V_x$).





\begin{figure}[H] \centering
\includegraphics[width=0.5\linewidth]{natural_tab.pdf} 
\vspace{10px}
\caption{Natural response of $V_6$ as a function of time in the interval from [0,20] ms}
\label{fig:natural}
\end{figure} 

%point 4
\subsection{Forced solution for V6 with f=1000Hz}
Now we determine the forced solution for $V_6$, in the same time interval and with a frequency of 1kHz. For this, we will redo the nodal method but using impedances now, and considering that $V_s=1$, taken from the initial condition of the source. We can determine the complex amplitudes in all nodes with the following equations:

\begin {equation}
	Z = \frac{1}{j w C}
	\label{eq:Z}
\end{equation}

\begin {equation}
	\tilde{V}_s = 1
	\label{eq:vs}
\end{equation}

\begin {equation}
	\tilde{V}_0 = 0
	\label{eq:p1}
\end{equation}
\begin {equation}
	\tilde{V}_4 = \tilde{V}_7
	\label{eq:p2}
\end{equation}
\begin {equation}
	\tilde{V}_5 - \tilde{V}_8 = K_d \frac{\tilde{V}_0 - \tilde{V}_4}{R_6}
	\label{eq:p3}
\end{equation}
\begin {equation}
	\tilde{V}_1 - \tilde{V}_0 = \tilde{V}_s
	\label{eq:p4}
\end{equation}
\begin {equation}
	\frac{\tilde{V}_2-\tilde{V}_1}{R_1} + \frac{\tilde{V}_2 - \tilde{V}_5}{R_3} + \frac{\tilde{V}_2 - \tilde{V}_3}{R_2} = 0
	\label{eq:p5}
\end{equation}
\begin {equation}
	\frac{\tilde{V}_3-\tilde{V}_2}{R_2} - K_b(\tilde{V}_2-\tilde{V}_5)  = 0
	\label{eq:p6}
\end{equation}
\begin {equation}
	\frac{\tilde{V}_1-\tilde{V}_2}{R_1} + \frac{\tilde{V}_0-\tilde{V}_4}{R_6} + \frac{\tilde{V}_0-\tilde{V}_5}{R_4} = 0
	\label{eq:p7}
\end{equation}
\begin {equation}
	K_b(\tilde{V}_2-\tilde{V}_5) + \frac{\tilde{V}_6-\tilde{V}_5}{R_5} + \frac{\tilde{V}_6-\tilde{V}_8}{Z}  = 0
	\label{eq:p8}
\end{equation}
\begin {equation}
	\frac{\tilde{V}_4-\tilde{V}_0}{R_6} + \frac{\tilde{V}_7 - \tilde{V}_8}{R_7} = 0
	\label{eq:p9}
\end{equation}
\begin {equation}
	\tilde{V}_x = \tilde{V}_6 - \tilde{V}_8
	\label{eq:p10}
\end{equation}


The complex amplitudes of the phasors are presented in  \textbf{Table ~\ref{tab:equivalent resistor}}
\begin{table}[H]
  \centering
  \begin{tabular}{|l|r|}
    \hline    
    {\bf Name} & {\bf Complex amplitude voltage}\\ \hline
    \input{phasor_tab}
  \end{tabular}
  \vspace{10px}
  \caption{Complex amplitudes in all nodes in Volts}
  \label{tab:equivalent resistor}
\end{table}
\vspace{10cm}

\pagebreak
%point 5
\subsection{Final total solution V6(t)}
In this section the final total solution $V_6$ for a frequency of 1KHz is determined using the natural and forced solutions determined in previous sections ($V_6$=$V_{n6}$+$V_{6f}$). In {Figure:~\ref{fig:theo5}} the voltage of the independent source $V_{s}$ and the voltage of $V_{6}$ were plotted for the time interval of [-5,20] ms. 
\begin{figure}[H] \centering
\includegraphics[width=0.60\linewidth]{theo5_tab.pdf}
\vspace{10px}
\caption{Voltage of $V_{6}(t)$ and $V_{s}(t)$ as functions of time from [-5,20] ms}
\label{fig:theo5}
\end{figure}

\subsection{Frequency responses $v_c(f)$, $v_s(f)$ and $v_6(f)$ for frequency range 0.1 Hz to 1 MHz}
\label{ref}
For this section, it was considered that $v_s( t) = sin( 2 \pi f t )$.
This circuit will act aproximatly as a low pass-filter. This means when the frequency is low the capacitor has a long time to charge up, then acting as an open circuit, and achieving praticly the same voltage as the input, therefore, there is a considerable voltage drop between node 6 and 8 and the capacitor will remane almost in phase with the voltage source.\par For high frequences the oposite is seen. The capacitor will have no time to charge before the input changes direction, so it will act as a short-circuit with almost no potential drop. Therefore, the capacitor will end up having a very diferent phase from the imput source as can be confirmed in the graphs bellow. This will be noted for frequences greater then the cutoff frequence ($f_c = \frac{1}{2.\pi.\tau}$). With our data $f_c$ is close to 50Hz, this is also the section of the graphs where we can see a drop in potencial for the capacitor and a big difference in phase start to show up.
Simplifying this circuit to something close to its Thevenan equivalent, using only $V_{eq}$, $R_{eq}$ and a Capacitor leads to the following equations, allowing for a better understanding of the phase and magnitude decrease as the frequency increases:

\begin{equation}
  V_c = \frac{V_s}{\sqrt{1 + (R_{eq}.C.2\pi.f)^2}}
  \label{eq:freqresp1}
\end{equation}

\begin{equation}
  \phi_{V_c} = -\frac{\pi}{2} + arctan(R_{eq}.C.2\pi.f)
  \label{eq:freqresp2}
\end{equation}
\begin{figure}[H] \centering
\includegraphics[width=0.8\linewidth]{freq_resp_tab.pdf}
\vspace{10px}
\caption{Graph for amplitude frequency response, in dB, of $V_c$, $V_6$ and $V_s$ for frequencies ranging from 0.1Hz to 1MHz (logarithmic scale).}
\label{fig:freq_resp}
\end{figure}



\begin{figure}[H] \centering
\includegraphics[width=0.8\linewidth]{angle_tab.pdf}
\vspace{10px}
\caption{Graph for the phase response, in degrees of $V_c$, $V_6$ and $V_s$ for frequencies ranging from 0.1Hz to 1MHz, displayed in a logarithmic scale. The phase of $V_6$ is actually continuous, and the reason for the apparent discontinuity in the graph is because of the domain of the arctan function - $D_{arctan}$ = ]-180, 180].}
\label{fig:angle_resp}
\end{figure}



