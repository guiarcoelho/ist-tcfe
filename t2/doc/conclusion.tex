\section{Conclusion}
\label{sec:conclusion}

The objective of this Laboratory of analysing a simple circuit, consisting only of voltage and current sources as well as resistors has been concluded. We successfully made both the theoretical and practical analysis of the circuit T1 and were now expecting to take some conclusions, calculate errors and show interesting data regarding the work we have just finished. Unfortunetly, our attempt to calculate any sort of errors was stopped by the inexistence of any. Both the theoretical and practical attempts have a perfect match of values of current and voltage in any studied element of the circuit. So if we can't take conclusions about errors, we can at least think of a reason to why there are none.
Well, as this laboratory was made with a simulated software machine, and the circuit is very simple it is easy to see why there would be no errors. The circuit is pretty straightforward, it doesn't have any components that depend on time that could lead to aproximate models for calculating any quantaty, so it eliminates any kind of error by aproximation. 
Also, the ngspice software uses perfect resistors, as well as wires with no resistence to connect the components, and voltage sources with no internal resistance, so that there is no unwanted Joule Efect in the circuit, as it would be a factor in a real experiment in laboratory. So, all the mistakes that could exist would have to be made by us.
This explanation ends our conclusion and also our laboratory T1 report.
% state the learning objective 


