\section{Conclusion}
\label{sec:conclusion}

As a conclusion, we can state that, unlike the previous two lab assignments, there is a degree of similarity between both analysis, in terms of precision. This was rather surprising, for we are dealing with a high-complexity component - the OPAMP - at least in terms of architecture. But, when we analyse it in a macroscopical point of view, it's rather predictable and stable - at least in this circuit. 

Overall, the laboratory gave us the opportunity to deepen our knowledge with OPAMPs and Active 2nd-order filters. We understood it as an extension of the second lab, where we analysed 1st-order RC circuits. 

It is also worth noting that the concept of the merit figure proves itself to be extremely important, because it narrows the gap between an academic point of view and an industrial/engineering approach. It made us understand that there are many factors involved in the construction of a circuit, some of them being the cost, the space availability and the customers' needs. Another limiting factor was the fixed values and quantities for each of the components, which made us have to think more thoroughly and inventively about how and where we would use them. A complete circuit that takes into account all these factors is most of the times really hard to achieve, which leads us to sacrifice some characteristics in favour of some more important others, achieving a balance between them all.

