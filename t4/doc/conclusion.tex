\section{Final Conclusion and General Notes}
\label{sec:conclusion}
 To conclude this assignement, we can say that simillar to the previous assignment, there is not a major degree of similarity between both analysis, in regards to precision. This was predicteble due to the fact that we are dealing with a non.linear circuit and the model used by \textit{Ngspice} is way more complex than the theoretical model used - the Incremental analysis one, presented on classes, only includes 2 resistors and a dependent current source. Altough these may be differente, the theoretical model gives an overview of the behaviour, so it is very useufl if we dont have the simulation tools available, or to very quickly verify the simulation results obtained.

With all of this present, this laboratory assigment as given us the opportunity ti deepen our knowledge regarding BJTs and how we can implement them to develop circuits with different purposes - in our case, an AUDIO Amplifier, even though the real model amplifiers are far more complex than the circuit implemented, achieving gains of around 115 dB. On the other hand, we were able to use new concepts such as the \textit{Time constant method}, the incremental models, the input and output impedances and the gain (these, eventough already familiar, allowed us to gain flexibility and easiness).

Observing our results, and especially the simulation ones, our main goal was to have a high gain and a large enough bandwidth that would cover at least 20Hz to 20kHz, which are the correspondent values to the human hearing range. We can say that the results observed more than cover the range mentioned before, and would be suitable for a real audio amplifier. Knowing this, one way we could improve our circuit would be to increase the gain even more, which it's not in our power to do.

