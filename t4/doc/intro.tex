\section{Introduction}
\label{sec:introduction}
In this laboratory assignment, we analysed, using both a theoretical and simulation analysis, a Sound Amplifier made of Bipolar Junction Transistors. It was made up of two different stages, a Gain stage, with the goal of having the maximum gain possible, and an output stage, whose goal was to lower the impendance. The circuit is presented in \ref{fig:circuitol3}.

As mentioned above, we analysed the circuit theoreticaly, combining Operating Point, allowing us to derive important values used in the incremental analysis.

Simultaneously, the circuit is analysed by computational simulation tools, via \textit{Ngspice}, and the results are compared to the theoretical results obtained, in Section \ref{sec:analysis}. The conclusions of this study are outlined in Section \ref{sec:conclusion}.

We also used Ngspice to analyse the circuit by computacional tools, and then compared with the results obtained in \ref{sec:analysis}. The results of this comparison are outlined in \ref{sec:conclusion}.



\begin{figure}[h] \centering
\includegraphics[width=1\linewidth]{circuitl4.pdf}
\caption{BJT Amplifier}
\label{fig:circuitol4}
\end{figure}

\clearpage
